\documentclass[a4paper,10pt]{article}
\usepackage[utf8]{inputenc}
\usepackage[default, osfigures, scale=0.95]{opensans}
\usepackage[T1]{fontenc}
\usepackage{lmodern}
\usepackage[french]{babel}
\usepackage[colorlinks, linkcolor=black]{hyperref}
\usepackage{fullpage}
\usepackage{verbatim}

\begin{document}
  \title{4) Boucles}
  \author{Programmation C (2013-2014)}
  \date{Guillaume Claret}
  \maketitle
  
  Les boucles permettent de répéter un séquence d'instructions. On indente dès que l'on rentre dans une boucle.
  
  \section{Boucle \texttt{while}}
  Répète une suite d'instructions tant qu'une condition est vraie. C'est la forme de boucle la plus générale.
  \begin{verbatim}
while (condition) {
  ... // code à exécuter tant que la condition est vraie
}
  \end{verbatim}
  De même que pour le \texttt{if}, on peut omettre les accolades si le code ne contient qu'une seule instruction~:
  \begin{verbatim}
while (condition)
  ...; // une seule ligne
  \end{verbatim}
  Exemple redemandant un nombre à l'utilisateur tant que celui-ci n'est plus grand que $10$~:
  \begin{verbatim}
int n = 0;
while (n < 10) {
  printf("Entrez un nombre supérieur à 10 : ");
  scanf("%d", &n);
}
printf("n = %d\n", n); // affiche un entier supérieur à 10
  \end{verbatim}
  Exemple de boucle infinie~:
  \begin{verbatim}
while (1)
  printf("Infini ...\n"); // affiche en boucle ce message
  \end{verbatim}
  
  \section{Boucle \texttt{do}-\texttt{while}}
  Similaire à la boucle \texttt{while}, mais s'exécute au moins un tour car teste la condition à la fin plutôt qu'au début~:
  \begin{verbatim}
do {
  ... // code à exécuter tant que la condition est vraie
} while (condition); // exception de syntaxe : toujours mettre un point-virgule ici
  \end{verbatim}
  
  \section{Boucle \texttt{for}}
  Forme de boucle spécialisée pour les énumérations~:
  \begin{verbatim}
for (initialisation; condition; action à chaque fin de tour) {
  ... // instructions à répéter
}
  \end{verbatim}
  Exemple typique~:
  \begin{verbatim}
for (i = 0; i <= 9; i++)
  printf("%d\n", i); // affiche tous les chiffres de 0 à 9
  \end{verbatim}
  
  \section{Instructions \texttt{break} et \texttt{continue}}
  L'instruction \texttt{break} permet de sortir prématurément d'une boucle~:
  \begin{verbatim}
int n = 0;
while (1) {
  if (n == 10)
    break; // le break sort de ce while infini
  printf("message\n"); // ce message n'est affiché que 10 fois
  n++;
}
  \end{verbatim}
  L'instruction \texttt{continue} permet de passer directement au tour de boucle suivant~:
  \begin{verbatim}
do {
  printf("Entrez n et m positifs de somme inférieure à 10 : ");
  scanf("%d", &n);
  if (n > 10)
    continue; // si n est déjà plus grand que 10, on peut directement tout redemmander
  scanf("%d", &m);
} while (n + m > 10);
  \end{verbatim}
  
  \section{Équivalences}
  \begin{tabular}{c|c|c}
    Version 1 & Version 2 & Version 3\\
    \hline
    \begin{minipage}[t]{4cm}
      \begin{verbatim}
do {
  instr;
} while (condition);
      \end{verbatim}
    \end{minipage} &
    \begin{minipage}[t]{4cm}
      \begin{verbatim}
instr;
while (condition) {
  instr;
}
      \end{verbatim}
    \end{minipage} &
    \begin{minipage}[t]{4cm}
      \begin{verbatim}
while (1) {
  instr;
  if (condition)
    break;
}
      \end{verbatim}
    \end{minipage}\\
    \hline
    \begin{minipage}[t]{4cm}
      \begin{verbatim}
for (init; cond; fin)
  instr;
      \end{verbatim}
    \end{minipage} &
    \begin{minipage}[t]{4cm}
      \begin{verbatim}
init;
while (cond) {
  instr;
  final;
}
      \end{verbatim}
    \end{minipage} &
    \begin{minipage}[t]{4cm}
      \begin{verbatim}
init;
while (1) {
  if (cond)
    break;
  instr;
  final;
}
      \end{verbatim}
    \end{minipage}\\
    \hline
    \begin{minipage}[t]{4cm}
      \begin{verbatim}
while (condition);
      \end{verbatim}
    \end{minipage} &
    \begin{minipage}[t]{4cm}
      \begin{verbatim}
while (condition)
  ;
      \end{verbatim}
    \end{minipage} &
    \begin{minipage}[t]{4cm}
      \begin{verbatim}
// code vide
      \end{verbatim}
    \end{minipage}
  \end{tabular}
\end{document}
