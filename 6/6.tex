\documentclass[a4paper,10pt]{article}
\usepackage[utf8]{inputenc}
\usepackage[default, osfigures, scale=0.95]{opensans}
\usepackage[T1]{fontenc}
\usepackage{lmodern}
\usepackage[french]{babel}
\usepackage[colorlinks, linkcolor=black]{hyperref}
\usepackage{fullpage}
\usepackage{verbatim}
\usepackage{tikz}

\tikzset{value/.style={draw, minimum height=0.7cm, minimum width=1cm}}
\tikzset{var/.style={below=0.4cm}}

\begin{document}
  \title{6) Fonctions}
  \author{Programmation C (2013-2014)}
  \date{Guillaume Claret}
  \maketitle
  
  Les fonctions permettent de mieux organiser un programme, en le découpant morceaux de code spécialisés et réutilisables.
  
  \section{Définition}
  
  \section{Passage de paramètres par référence}
\end{document}
