\documentclass[a4paper,10pt]{article}
\usepackage[utf8]{inputenc}
\usepackage[default, osfigures, scale=0.95]{opensans}
\usepackage[T1]{fontenc}
\usepackage{lmodern}
\usepackage[french]{babel}
\usepackage[colorlinks, linkcolor=black]{hyperref}
\usepackage{fullpage}
\usepackage{verbatim}
\usepackage{tikz}

\tikzset{value/.style={draw, minimum height=0.7cm, minimum width=1cm}}
\tikzset{var/.style={below=0.4cm}}

\begin{document}
  \title{5) Pointeurs}
  \author{Programmation C (2013-2014)}
  \date{Guillaume Claret}
  \maketitle
  
  Les pointeurs représentent des adresses mémoires. Ils seront surtout utilisés pour manipuler certaines structures de données comme les tableaux, ou pour le passage d'arguments par référence.
  
  \section{Variables et mémoire}
  Un programme réside entièrement en mémoire vive et ne peut accéder aux fichiers que via des fonctions dédiées (comme \texttt{fopen()} en C). Une mémoire vive de taille $n$ est une liste d'octets numérotés de $0$ à $n - 1$ (un octet est une série de 8 bits). Le numéro d'un octet est appelé son \emph{adresse}. Par exemple, pour une mémoire de 4 Go, les adresses vont de $0$ à~:
  \[
    4 * G - 1 = 4 * K^3 - 1 = 4 * (2^{10})^3 - 1 = 2^{32} - 1 = 4294967295
  \]
  (en informatique un kilo (K) vaut $2^{10} = 1024$ et non pas $1000$).
  
  En C on remplit la mémoire en déclarant des variables. Celles-ci sont attribuées à des adresses quelconques, en fonction de ce qui est encore disponible. Par exemple~:
  \begin{verbatim}
int a = 12, b = 3; char c = 18;
  \end{verbatim}
  alloue les trois cases suivantes~:
  \begin{center}
    \begin{tikzpicture}
      \node[value, minimum width=4cm] (a) at (0, 0) {$12$};
      \node[value, minimum width=4cm] (b) at (5, 0) {$3$};
      \node[value, minimum width=4cm] (c) at (10, 0) {$18$};
      \node[var] at (a) {\texttt{int a}};
      \node[var] at (b) {\texttt{int b}};
      \node[var] at (c) {\texttt{char c}};
    \end{tikzpicture}
  \end{center}
  Comme les \texttt{int} sont représentés par quatre octets et les \texttt{char} par un octet, on alloue plus exactement neuf octets~:
  \begin{center}
    \begin{tikzpicture}
      \node[value, minimum width=1cm] (a) at (0, 0) {};
      \node[value, minimum width=1cm] at (1, 0) {};
      \node[value, minimum width=1cm] at (2, 0) {};
      \node[value, minimum width=1cm] at (3, 0) {};
      \node[value, minimum width=1cm] (b) at (5, 0) {};
      \node[value, minimum width=1cm] at (6, 0) {};
      \node[value, minimum width=1cm] at (7, 0) {};
      \node[value, minimum width=1cm] at (8, 0) {};
      \node[value, minimum width=1cm] (c) at (10, 0) {};
      \node[var] at (a) {\texttt{int a}};
      \node[var] at (b) {\texttt{int b}};
      \node[var] at (c) {\texttt{char c}};
    \end{tikzpicture}
  \end{center}
  On note \texttt{\&a} l'adresse du premier octet de \texttt{a}, et on l'appelle \emph{l'adresse} de \texttt{a}. Par convention l'adresse $0$, notée plus souvent \texttt{NULL}, représente une adresse invalide et est utilisée comme adresse bidon.
  
  \section{Définition et exemples}
  Un pointeur est une variable contenant une adresse. On les déclare en ajoutant une étoile au type des variables sur lequels elles pointent~:
  \begin{verbatim}
int * p;
  \end{verbatim}
  On initialise \texttt{p} en récupérant l'adresse d'une autre variable de type \texttt{int}~:
  \begin{verbatim}
p = &a;
  \end{verbatim}
  On obtient~:
  \begin{center}
    \begin{tikzpicture}
      \node[value, minimum width=2cm] (a) at (0, 0) {$12$};
      \node[value, minimum width=2cm] (b) at (3, 0) {$3$};
      \node[value, minimum width=2cm] (c) at (6, 0) {$18$};
      \node[value, minimum width=2cm] (p) at (9, 0) {\texttt{\& a}};
      \node[var] at (a) {\texttt{int a}};
      \node[var] at (b) {\texttt{int b}};
      \node[var] at (c) {\texttt{char c}};
      \node[var] at (p) {\texttt{int * p}};
      \draw[->, >=latex] (p.north) to[bend right] (a.north);
    \end{tikzpicture}
  \end{center}
  On affiche un pointeur avec \texttt{\%p}~:
  \begin{verbatim}
printf("p = %p\n", p);
  \end{verbatim}
  (c'est une adresse, donc généralement un nombre quelconque et assez grand). On peut modifier \texttt{p}, à condition de toujours pointer sur une donnée de type \texttt{int}~:
  \begin{verbatim}
p = &b;
  \end{verbatim}
  \begin{center}
    \begin{tikzpicture}
      \node[value, minimum width=2cm] (a) at (0, 0) {$12$};
      \node[value, minimum width=2cm] (b) at (3, 0) {$3$};
      \node[value, minimum width=2cm] (c) at (6, 0) {$18$};
      \node[value, minimum width=2cm] (p) at (9, 0) {\texttt{\& b}};
      \node[var] at (a) {\texttt{int a}};
      \node[var] at (b) {\texttt{int b}};
      \node[var] at (c) {\texttt{char c}};
      \node[var] at (p) {\texttt{int * p}};
      \draw[->, >=latex] (p.north) to[bend right] (b.north);
    \end{tikzpicture}
  \end{center}
  Pour lire la valeur pointée par \texttt{p} (déréférencement) on utilise aussi l'étoile (ce symbole a donc deux significations, typage et déréférencement)~:
  \begin{verbatim}
printf("%d\n", *p);
  \end{verbatim}
  affiche la valeur de \texttt{b}, c'est-à-dire $3$. On peut aussi directement modifier \texttt{b} en passant par \texttt{p}~:
  \begin{verbatim}
*p = 23;
  \end{verbatim}
  \begin{center}
    \begin{tikzpicture}
      \node[value, minimum width=2cm] (a) at (0, 0) {$12$};
      \node[value, minimum width=2cm] (b) at (3, 0) {$23$};
      \node[value, minimum width=2cm] (c) at (6, 0) {$18$};
      \node[value, minimum width=2cm] (p) at (9, 0) {\texttt{\& b}};
      \node[var] at (a) {\texttt{int a}};
      \node[var] at (b) {\texttt{int b}};
      \node[var] at (c) {\texttt{char c}};
      \node[var] at (p) {\texttt{int * p}};
      \draw[->, >=latex] (p.north) to[bend right] (b.north);
    \end{tikzpicture}
  \end{center}
  Enfin, les pointeurs peuvent eux-mêmes pointer sur des pointeurs~:
  \begin{verbatim}
int ** p2 = &p;
  \end{verbatim}
  \begin{center}
    \begin{tikzpicture}
      \node[value, minimum width=2cm] (a) at (0, 0) {$12$};
      \node[value, minimum width=2cm] (b) at (3, 0) {$23$};
      \node[value, minimum width=2cm] (c) at (6, 0) {$18$};
      \node[value, minimum width=2cm] (p) at (9, 0) {\texttt{\& b}};
      \node[value, minimum width=2cm] (p2) at (12, 0) {\texttt{\& p}};
      \node[var] at (a) {\texttt{int a}};
      \node[var] at (b) {\texttt{int b}};
      \node[var] at (c) {\texttt{char c}};
      \node[var] at (p) {\texttt{int * p}};
      \node[var] at (p2) {\texttt{int ** p2}};
      \draw[->, >=latex] (p.north) to[bend right] (b.north);
      \draw[->, >=latex] (p2.north) to[bend right] (p.north);
    \end{tikzpicture}
  \end{center}
\end{document}
