\documentclass[a4paper,10pt]{article}
\usepackage[utf8]{inputenc}
\usepackage[default, osfigures, scale=0.95]{opensans}
\usepackage[T1]{fontenc}
\usepackage{lmodern}
\usepackage[french]{babel}
\usepackage[colorlinks, linkcolor=black]{hyperref}
\usepackage{fullpage}
\usepackage{verbatim}
\usepackage{tikz}

\begin{document}
  \title{5) Pointeurs}
  \author{Programmation C (2013-2014)}
  \date{Guillaume Claret}
  \maketitle
  
  Les pointeurs représentent des adresses mémoires. Ils seront surtout utilisés pour manipuler certaines structures de données comme les tableaux, ou pour le passage d'arguments par référence.
  
  \section{Variables et mémoire}
  \tikzset{var/.style={draw, minimum height=0.7cm, minimum width=2cm}}
  \begin{tikzpicture}
    \node[var] at (0, 0) {12};
    \node[var] at (3, 0) {3};
    \node[var] at (6, 0) {18};
  \end{tikzpicture}
  
  \section{Définition et exemples}
  
\end{document}
