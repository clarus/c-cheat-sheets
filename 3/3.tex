\documentclass[a4paper,10pt]{article}
\usepackage[utf8]{inputenc}
\usepackage[default, osfigures, scale=0.95]{opensans}
\usepackage[T1]{fontenc}
\usepackage{lmodern}
\usepackage[french]{babel}
\usepackage[colorlinks, linkcolor=black]{hyperref}
\usepackage{fullpage}
\usepackage{verbatim}
\usepackage{tabu}

\begin{document}
  \title{3) Tests et booléens}
  \author{Programmation C (2013-2014)}
  \date{Guillaume Claret}
  \maketitle
  
  \section{Tests}
  \subsection{If, then, else}
  L'instruction \texttt{if} permet de réaliser un test. La condition est toujours entre parenthèses~:
  \begin{verbatim}
if (condition) {
  ... // code à exécuter si la condition est vraie
} else {
  ... // code à exécuter si la condition est fausse
}
  \end{verbatim}
  La partie \texttt{else} est optionnelle~:
  \begin{verbatim}
if (condition) {
  ... // code à exécuter si la condition est vraie
}
  \end{verbatim}
  Si le code à exécuter ne contient qu'une seule instruction on peut omettre les accolades~:
  \begin{verbatim}
if (condition)
  printf(...); // code d'une seule instruction
  \end{verbatim}
  Exemple avec plusieurs \texttt{if} qui affiche «~n grand~»~:
  \begin{verbatim}
int n = 12;
if (n < 0)
  printf("n négatif\n");
else if (n < 10)
  printf("n petit\n");
else
  printf("n grand\n");
  \end{verbatim}
  
  \newpage
  \subsection{Switch, case, default}
  Le \texttt{switch} permet de comparer rapidement une valeur avec un ensemble fini de possibilités~:
  \begin{verbatim}
switch(valeur) {
case solution_1:
  ... // code à exécuter si la valeur = solution_1
  break;
...
case solution_n:
  ... // code à exécuter si la valeur = solution_n
  break;
default: // le default est optionnel
  ... // code à exécuter sinon
}
  \end{verbatim}
  
  \subsection{Mini-test}
  On peut faire des tests dans une instruction avec l'opérateur \texttt{?:}~:
  \begin{verbatim}
condition ? valeur_si_vrai : valeur_si_faux
  \end{verbatim}
  Exemple affichant 1~:
  \begin{verbatim}
int n = 12;
printf("%d\n", n > 0 ? 1 : -1);
  \end{verbatim}
  
  \section{Booléens}
  Les booléens (valeurs vrai / faux) sont représentés par des entiers avec la convention~:
  \begin{itemize}
    \item $0$ vaut faux
    \item toute valeur non nulle vaut vrai (souvent on prend $1$)
  \end{itemize}
  
  \subsection{Opérateurs logiques}
  \begin{tabu}{cc|c|c|c}
    a & b & a \&\& b & a || b & !\,b\\
    \hline
    $0$ & $0$ & $0$ & $0$ & $1$\\
    $0$ & $1$ & $0$ & $1$ & $0$\\
    $1$ & $0$ & $0$ & $1$ &\\
    $1$ & $1$ & $1$ & $1$ &\\
  \end{tabu}
  
  \subsection{Comparaisons}
  \begin{tabu}{cc|cc|cc|cc}
    a & b & a == b & a != b & a < b & a <= b & a > b & a >= b\\
    \hline
    $12$ & $23$ & $0$ & $1$ & $1$ & $1$ & $0$ & $0$\\
    $12$ & $12$ & $1$ & $0$ & $0$ & $1$ & $0$ & $1$\\
    $23$ & $12$ & $0$ & $1$ & $0$ & $0$ & $1$ & $1$\\
  \end{tabu}
\end{document}
