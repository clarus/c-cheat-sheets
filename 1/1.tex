\documentclass[a4paper,10pt]{article}
\usepackage[utf8]{inputenc}
\usepackage[default, osfigures, scale=0.95]{opensans}
\usepackage[T1]{fontenc}
\usepackage{lmodern}
\usepackage[french]{babel}
\usepackage[colorlinks, linkcolor=black]{hyperref}
\usepackage{fullpage}
\usepackage{verbatim}
\usepackage{tabu}

\begin{document}
  \title{1) Entrées-sorties et variables}
  \author{Programmation C (2013-2014)}
  \date{Guillaume Claret}
  \maketitle
  
  \section{Premier programme}
  \begin{verbatim}
// Un commentaire commence par //.
// On inclut les fonctions prédéfinies usuelles :
#include <stdio.h>

// "main" est toujours le nom de la fonction principale :
int main() {
  // Un message de bienvenue. Les instructions se terminent par un ";".
  printf("Bonjour.\n");
  
  // On fini le programme en retournant la valeur par défaut 0 :
  return 0;
}
  \end{verbatim}
  
  \section{Compilation}
  \begin{description}
    \item[Écriture] On crée un fichier \texttt{ex1.c} que l'on ouvre avec \texttt{gedit} (ou un autre éditeur) puis on écrit le programme.
    \item[Compilation] À la ligne de commande (\texttt{-Wall} active tous les warnings, \texttt{ex1} est le nom de l'exécutable à générer) :
      \begin{verbatim}
gcc -Wall ex1.c -o ex1
      \end{verbatim}
    \item[Lancement] À la ligne de commande :
      \begin{verbatim}
./ex1
      \end{verbatim}
  \end{description}
  
  \section{Variables}
  Les variables permettent de stocker des valeurs et de les modifier.
  \begin{description}
    \item[Déclaration] Un type suivit d'une liste de noms. Les noms de variables contiennent des lettres sans accents, des chiffres ou des underscores (symbole \texttt{\_}). Ils ne peuvent pas commencer par un chiffre. On peut donner une valeur initiale.
      \begin{verbatim}
int var1, x = 12, n;
      \end{verbatim}
    \item[Types] Quelques types standards :\\
      \begin{tabu}{c|c|c|c}
        Type & Taille\footnote{Peut dépendre de l'architecture, valeurs pour un PC 64 bits sous Linux.} & Contenu & Intervalle\\
        \hline
        \texttt{char} & 8 bits & entier & $[- 128; 127]$\\
        \texttt{short} & 16 bits & entier & $[-32.768; 32 .67]$\\
        \texttt{int} & 32 bits & entier & $[- 2^{31}; 2^{31} - 1]$\\
        \texttt{long} & 64 bits & entier & $[- 2^{63}; 2^{63} - 1]$\\
        \hline
        \texttt{unsigned char} & 8 bits & entier & $[0; 255]$\\
        \texttt{unsigned short} & 16 bits & entier & $[0; 65.535]$\\
        \texttt{unsigned int} & 32 bits & entier & $[0; 2^{32} - 1]$\\
        \texttt{unsigned long} & 64 bits & entier & $[0; 2^{64} - 1]$\\
        \hline
        \texttt{float} & 32 bits & flottant & $3,4 \times 10^{-38}$ à $3,4 \times 10^{38}$\\
        \texttt{double} & 64 bits & flottant & $1,7 \times 10^{-308}$ à $1,7 \times 10^{308}$
      \end{tabu}
    \item[Modification] La variable à modifier est à gauche. Pour ajouter 3 à \texttt{n} :
      \begin{verbatim}
n = n + 3;
      \end{verbatim}
  \end{description}
  
  \section{Entrées-sorties}
  \begin{description}
    \item[Lire] Avec \texttt{printf} suivit du message entre guillemets. Le \texttt{\textbackslash n} représente un retour à la ligne :
      \begin{verbatim}
printf("Bonjour.\n");
      \end{verbatim}
      Pour afficher le contenu d'une variable entière :
      \begin{verbatim}
printf("Valeur de x : %d\n", x);
      \end{verbatim}
    \item[Écrire] Avec \texttt{scanf} suivit d'un format et de variables précédées par des \texttt{\&}. Pour lire deux entiers :
      \begin{verbatim}
scanf("%d%d", &n1, &n2);
      \end{verbatim}
    \item[Codes spéciaux] Certains codes n'ont de sens que pour \texttt{printf} ou que pour \texttt{scanf} :\\
      \begin{tabu}{c|l}
        Code & Signification\\
        \hline
        \texttt{\textbackslash n} & retour à la ligne\\
        \texttt{\textbackslash t} & tabulation\\
        \texttt{\textbackslash "} & \texttt{"}\\
        \texttt{\textbackslash \textbackslash} & \texttt{\textbackslash}\\
        \hline
        \texttt{\%d} & entier\\
        \texttt{\%u} & entier non-signé\\
        \texttt{\%f} & flottant\\
        \texttt{\%.2f} & flottant avec 2 chiffres après la virgule\\
        \texttt{\%8f} & flottant affiché avec une largeur d'au moins 8 caractères\\
        \texttt{\%c} & caractère\\
        \texttt{\%*c} & caractère lu puis oublié (utile pour manger des retours à la ligne)\\
        \texttt{\%\%} & \texttt{\%}\\
      \end{tabu}
  \end{description}
\end{document}
