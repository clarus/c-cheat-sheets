\documentclass[a4paper,11pt]{article}
\usepackage[utf8]{inputenc}
\usepackage[default, osfigures, scale=0.95]{opensans}
\usepackage[T1]{fontenc}
\usepackage{lmodern}
\usepackage[french]{babel}
\usepackage[colorlinks, linkcolor=black]{hyperref}
\usepackage{verbatim}

\begin{document}
  \title{Entrées-sorties et variables}
  \author{Programmation C}
  \date{}
  \maketitle
  
  \section{Premier programme}
  \begin{verbatim}
// Un commentaire commence par //.
// On inclut les fonctions prédéfinies usuelles :
#include <stdio.h>

// "main" est toujours le nom de la fonction principale :
int main() {
  // Un message de bienvenue :
  printf("Bonjour.\n");
  
  // On termine en retournant par défaut la valeur 0 :
  return 0;
}
  \end{verbatim}
  
  \section{Compilation}
  \begin{description}
    \item[Écriture] On crée un fichier \texttt{ex1.c} que l'on ouvre avec \texttt{gedit} (ou un autre éditeur) puis on écrit le programme.
    \item[Compilation] À la ligne de commande (\texttt{-Wall} active tous les warnings, \texttt{ex1} est le nom de l'exécutable à générer) :
      \begin{verbatim}
gcc -Wall ex1.c -o ex1
      \end{verbatim}
    \item[Lancement] À la ligne de commande :
      \begin{verbatim}
./ex1
      \end{verbatim}
  \end{description}
  
  \section{Variables}
  
  \section{Entrées-sorties}
\end{document}
